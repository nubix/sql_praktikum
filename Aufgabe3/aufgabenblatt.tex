\documentclass[11pt,a4paper,DIV=9]{scrartcl}
\usepackage{pgf}
\usepackage{tikz}
\usetikzlibrary{arrows,automata,positioning,shadows}
\usepackage{ngerman}
\usepackage[utf8]{inputenc}
\usepackage{amsmath,amssymb}
\usepackage{tikz-er2}
\usepackage{enumerate}
\usepackage{listings}
\usetikzlibrary{shapes,snakes}
% Schriftart ändern
\renewcommand{\rmdefault}{ppl}
%Möglichkeit zur Änderung von Überschriften
\usepackage{sectsty}
%Überschrift \section uandern
\definecolor{blue}{RGB}{76 , 92, 153}
\allsectionsfont{\color{blue}}
\paragraphfont{\color{blue}}

%Variable Blattnummer
\newcommand{\blatt}[1]{
  \newcommand{\blattnr}{#1}
}
%Aufgabe und Aufgabenteil definieren
\newcounter{temp}
\newcommand{\aufgabe}[1]{
  \setcounter{temp}{\value{subsection}}
  \setcounter{subsection}{#1}
  \addtocounter{subsection}{-1}
  \subsection{Aufgabe}
  \setcounter{subsection}{\value{temp}}
}
\newcommand{\teil}[2][]{
  \subsubsection*{#2) #1}
}
\renewcommand{\author}[1]{\renewcommand{\author}{#1}}
\renewcommand{\title}[1]{\renewcommand{\title}{#1}}
\newcommand{\makehomeworktitle}{
  \begin{minipage}[t]{6.5cm}
    \sf{\author}
  \end{minipage}
  \begin{minipage}[t]{6.5cm}
    \begin{flushright}
      \sf{\title\\\today}
    \end{flushright}
  \end{minipage}
  \\[0.2cm]
  \begin{center}
    \sf{
      \color{blue}{
        \LARGE{NET-FIRST! -- Aufgabenblatt \blattnr}
      }
    }
  \end{center}
  \vspace{0.1cm}
}

%%%%%%%%%%%%%%%%%%%%%%%%
%%% Statisch
\author{{[}4131658{]} Jan Germann \\{[}4054962{]} Christian Ratz\\Übungsgruppe 1}
\title{SQL Praktikum}

%%% Auf jedes Hausaufgabenblatt anpassen
\blatt{3}
%%%%%%%%%%%%%%%%%%%%%%%%
\setcounter{section}{\blattnr}

\definecolor{dkgreen}{rgb}{0,0.6,0}
\definecolor{gray}{rgb}{0.5,0.5,0.5}
\definecolor{mauve}{rgb}{0.58,0,0.82}

\lstset{ 
  basicstyle=\footnotesize\ttfamily,
  language=sql,                % the language of the code
  numbers=left,
  numberstyle=\tiny\color{gray},
  keywordstyle=\color{blue},          % keyword style
  commentstyle=\color{dkgreen},       % comment style
  stringstyle=\color{mauve}}

\begin{document}
\makehomeworktitle
  \aufgabe{1}
   Entry(\underline{id}, name, production\_year, type $ \rightarrow $ Type) \\
   Person(\underline{id}, name) \\
   Type(\underline{id}, name, description) \\
   Task(\underline{id}, name) \\
   participates(\underline{person} $ \rightarrow $ Person,
   \underline{task} $ \rightarrow $ Task,
   \underline{entry} $\rightarrow$ Entry) \\\\
   \aufgabe{2}
   \begin{enumerate}
   \item \textbf{Entry}
   % funktioniert!
\begin{lstlisting}
CREATE TABLE Entry (
  id VARCHAR(400) NOT NULL,
  name VARCHAR(400),
  production_year INTEGER,
  type VARCHAR(10) REFERENCES Type(id),
  PRIMARY KEY(id)
);
\end{lstlisting}
    \item \textbf{Person}
    % Funktioniert!
\begin{lstlisting}
CREATE TABLE Person (
  id VARCHAR(400) NOT NULL, 
  name VARCHAR(400), 
  PRIMARY KEY(id)
);
    \end{lstlisting}
    \item \textbf{Type}
    % Funktioniert !
\begin{lstlisting}
CREATE TABLE Type (
  id VARCHAR(10) NOT NULL, 
  name VARCHAR(400), 
  description VARCHAR(400),
  PRIMARY KEY(id)
);
\end{lstlisting}
    \item \textbf{Task}
    % Funktioniert ! 
\begin{lstlisting}
CREATE TABLE Task (
  id INTEGER NOT NULL, 
  name VARCHAR(400),
  PRIMARY KEY(id)
);
\end{lstlisting}
    \item \textbf{participates}
    % Funktioniert !
\begin{lstlisting}
CREATE TABLE participates (
  person VARCHAR(400) NOT NULL REFERENCES Person(id),
  task INTEGER NOT NULL REFERENCES Task(id),
  entry VARCHAR(400) NOT NULL REFERENCES Entry(id),
  PRIMARY KEY(person, entry, task)
);
\end{lstlisting}	
    \end{enumerate}
\aufgabe{3}

\begin{enumerate}[\textbf{Abfrage} 1:]

\item Finde den \textit{title} und \textit{type} aller \textit{Movies} seit 2004. 
\begin{lstlisting} 
SELECT TITLE_TITLE, TITLE_TYPE
FROM IMDB.MOVIES
WHERE TITLE_YEAR >= 2004;
\end{lstlisting}

\item Finde die Name aller \textit{Actors} (und \textit{Actresses}), die in mindestens einem der Filme aus Anfrage 1 mitgespielt haben. Gib jeden Namen nur einmal aus!
\begin{lstlisting}
SELECT DISTINCT NAME_NAME
FROM IMDB.ACTORS a, IMDB.MOVIES m WHERE a.TITLE_ID = m.TITLE_ID
AND m.TITLE_YEAR >= 2004
UNION
SELECT DISTINCT NAME_NAME
FROM IMDB.ACTRESSES a, IMDB.MOVIES m WHERE a.TITLE_ID = m.TITLE_ID
AND m.TITLE_YEAR >= 2004;
\end{lstlisting}

\item Finde heraus, wie viele \textit{Movies} es seit 2004 gegeben hat (die Anzahl)!
\begin{lstlisting}
SELECT COUNT(*)
FROM IMDB.MOVIES
WHERE TITLE_YEAR >= 2004
AND TITLE_TYPE = 'film';
\end{lstlisting}

\item Finde die maximale und durchschnittliche Anzahl von \textit{Movies}, in denen \textit{Actors} (und \textit{Actresses}) seit 2004 mitgespielt haben. \textit{Actors} (und \textit{Actresses}), die nur in einem \textit{Movie} mitgespielt haben, sollen nicht mit betrachtet werden!
\begin{lstlisting}
WITH
   Avrg AS (
	SELECT AVG(COUNT(*)) AS average 
	FROM IMDB.MOVIES 
	WHERE m.TITLE = a.TITLE		
	AND m.TITLE_YEAR >= 2004)
(SELECT MAX(c.Count) AS max
FROM IMDB.MOVIES m,
  	 IMDB.ACTORS a,
   (SELECT COUNT(*) AS Count FROM IMDB.MOVIES) c
	WHERE m.TITLE_ID = a.TITLE_ID
	AND TITLE_YEAR >= 2004
UNION
SELECT MAX(c.Count) AS max
FROM IMDB.MOVIES m,
	 IMDB.ACTRESSES a,
	(SELECT COUNT(*) AS Count FROM IMDB.MOVIES) c
	 WHERE m.TITLE_ID = a.TITLE_ID
	 AND TITLE_YEAR >= 2004)
\end{lstlisting}

\item Finde die \textit{names} aller \textit{Actors} (und \textit{Actresses}), die in mindestens 30 \textit{Movies} (nicht \textit{Serien}) seit 2004 mitgespielt haben und die Anzahl der \textit{Movies} in denen sie mitgespielt haben, absteigend sortiert nach der Anzahl der \textit{title} und ihre \textit{names} in alphabetischer Reihenfolge.
\begin{lstlisting}
\end{lstlisting}
\end{enumerate}

\aufgabe{4}
\begin{lstlisting}



INSERT INTO Type (id, name)
VALUES ('film', 'Ein Film');

INSERT INTO Entry (id, name, production_year, type)
 (SELECT TITLE_ID,
         TITLE_TITLE,
         TITLE_YEAR,
         TITLE_TYPE
  FROM IMDB.MOVIES
  WHERE TITLE_YEAR = 2001
  AND   TITLE_TYPE = 'film');

INSERT INTO Task (id, name)
VALUES  (1, 'Actor'),
        (2, 'Actress'),
        (3, 'Writer'),
        (4, 'Director'),
        (5, 'Cinematographer'),
        (6, 'Composer'),
        (7, 'Costume_Designer'),
        (8, 'Producer'),
        (9, 'Editor');

INSERT INTO DBLAB01.Person (id, name)
   (SELECT DISTINCT a.NAME_ID, a.NAME_NAME
   FROM IMDB.ACTORS a, DBLAB01.Entry
   WHERE a.TITLE_ID = Entry.id);
INSERT INTO participates (person, task, entry)
   (SELECT NAME_NAME, 1, TITLE_ID
   FROM IMDB.ACTORS, Entry
   WHERE TITLE_ID = Entry.id);

INSERT INTO DBLAB01.Person (id, name)
   (SELECT DISTINCT a.NAME_ID, a.NAME_NAME
   FROM IMDB.ACTRESSES a, DBLAB01.Entry
   WHERE a.TITLE_ID = Entry.id);
INSERT INTO participates (person, task, entry)
   (SELECT NAME_NAME, 2, TITLE_ID
   FROM IMDB.ACTRESSES, Entry
   WHERE TITLE_ID = Entry.id);

INSERT INTO DBLAB01.Person (id, name)
   (SELECT DISTINCT a.NAME_ID, a.NAME_NAME
   FROM IMDB.WRITERS a, DBLAB01.Entry
   WHERE a.TITLE_ID = Entry.id);
INSERT INTO participates (person, task, entry)
   (SELECT NAME_NAME, 3, TITLE_ID
   FROM IMDB.WRITERS, Entry
   WHERE TITLE_ID = Entry.id);

INSERT INTO DBLAB01.Person (id, name)
   (SELECT DISTINCT a.NAME_ID, a.NAME_NAME
   FROM IMDB.DIRECTORS a, DBLAB01.Entry
   WHERE a.TITLE_ID = Entry.id);
INSERT INTO participates (person, task, entry)
   (SELECT NAME_NAME, 4, TITLE_ID
   FROM IMDB.DIRECTORS, Entry
   WHERE TITLE_ID = Entry.id);

INSERT INTO DBLAB01.Person (id, name)
   (SELECT DISTINCT a.NAME_ID, a.NAME_NAME
   FROM IMDB.CINEMATOGRAPHERS a, DBLAB01.Entry
   WHERE a.TITLE_ID = Entry.id);
INSERT INTO participates (person, task, entry)
   (SELECT NAME_NAME, 5, TITLE_ID
   FROM IMDB.CINEMATOGRAPHERS, Entry
   WHERE TITLE_ID = Entry.id);

INSERT INTO DBLAB01.Person (id, name)
   (SELECT DISTINCT a.NAME_ID, a.NAME_NAME
   FROM IMDB.COMPOSERS a, DBLAB01.Entry
   WHERE a.TITLE_ID = Entry.id);
INSERT INTO participates (person, task, entry) 
   (SELECT NAME_NAME, 6, TITLE_ID
   FROM IMDB.COMPOSERS, Entry
   WHERE TITLE_ID = Entry.id);

INSERT INTO DBLAB01.Person (id, name)
   (SELECT DISTINCT a.NAME_ID, a.NAME_NAME
   FROM IMDB.COSTUME_DESIGNERS a, DBLAB01.Entry
   WHERE a.TITLE_ID = Entry.id);
INSERT INTO participates (person, task, entry)
   (SELECT NAME_NAME, 7, TITLE_ID
   FROM IMDB.COSTUME_DESIGNERS, Entry
   WHERE TITLE_ID = Entry.id);

INSERT INTO DBLAB01.Person (id, name)
   (SELECT DISTINCT a.NAME_ID, a.NAME_NAME
   FROM IMDB.PRODUCERS a, DBLAB01.Entry
   WHERE a.TITLE_ID = Entry.id);
INSERT INTO participates (person, task, entry)
   (SELECT NAME_NAME, 8, TITLE_ID
   FROM IMDB.PRODUCERS, Entry
   WHERE TITLE_ID = Entry.id);

INSERT INTO DBLAB01.Person (id, name)
   (SELECT DISTINCT a.NAME_ID, a.NAME_NAME
   FROM IMDB.EDITORS a, DBLAB01.Entry
   WHERE a.TITLE_ID = Entry.id);
INSERT INTO participates (person, task, entry)
   (SELECT NAME_NAME, 9, TITLE_ID
   FROM IMDB.EDITORS, Entry
   WHERE TITLE_ID = Entry.id);
\end{lstlisting}
\end{document}

