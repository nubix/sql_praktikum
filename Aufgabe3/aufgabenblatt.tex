\documentclass[11pt,a4paper,DIV=9]{scrartcl}
\usepackage{pgf}
\usepackage{tikz}
\usetikzlibrary{arrows,automata,positioning,shadows}
\usepackage{ngerman}
\usepackage[utf8]{inputenc}
\usepackage{amsmath,amssymb}
\usepackage{tikz-er2}
\usepackage{enumerate}
\usepackage{listings}
\usetikzlibrary{shapes,snakes}
% Schriftart ändern
\renewcommand{\rmdefault}{ppl}
%Möglichkeit zur Änderung von Überschriften
\usepackage{sectsty}
%Überschrift \section uandern
\definecolor{blue}{RGB}{76 , 92, 153}
\allsectionsfont{\color{blue}}
\paragraphfont{\color{blue}}

%Variable Blattnummer
\newcommand{\blatt}[1]{
  \newcommand{\blattnr}{#1}
}
%Aufgabe und Aufgabenteil definieren
\newcounter{temp}
\newcommand{\aufgabe}[1]{
  \setcounter{temp}{\value{subsection}}
  \setcounter{subsection}{#1}
  \addtocounter{subsection}{-1}
  \subsection{Aufgabe}
  \setcounter{subsection}{\value{temp}}
}
\newcommand{\teil}[2][]{
  \subsubsection*{#2) #1}
}
\renewcommand{\author}[1]{\renewcommand{\author}{#1}}
\renewcommand{\title}[1]{\renewcommand{\title}{#1}}
\newcommand{\makehomeworktitle}{
  \begin{minipage}[t]{6.5cm}
    \sf{\author}
  \end{minipage}
  \begin{minipage}[t]{6.5cm}
    \begin{flushright}
      \sf{\title\\\today}
    \end{flushright}
  \end{minipage}
  \\[0.2cm]
  \begin{center}
    \sf{
      \color{blue}{
        \LARGE{NET-FIRST! -- Aufgabenblatt \blattnr}
      }
    }
  \end{center}
  \vspace{0.1cm}
}

%%%%%%%%%%%%%%%%%%%%%%%%
%%% Statisch
\author{{[}4131658{]} Jan Germann \\{[}4054962{]} Christian Ratz\\Übungsgruppe 1}
\title{SQL Praktikum}

%%% Auf jedes Hausaufgabenblatt anpassen
\blatt{3}
%%%%%%%%%%%%%%%%%%%%%%%%
\setcounter{section}{\blattnr}

\definecolor{dkgreen}{rgb}{0,0.6,0}
\definecolor{gray}{rgb}{0.5,0.5,0.5}
\definecolor{mauve}{rgb}{0.58,0,0.82}

\lstset{ 
  basicstyle=\footnotesize\ttfamily,
  language=sql,                % the language of the code
  numbers=left,
  numberstyle=\tiny\color{gray},
  keywordstyle=\color{blue},          % keyword style
  commentstyle=\color{dkgreen},       % comment style
  stringstyle=\color{mauve}}

\begin{document}
\makehomeworktitle
  \aufgabe{1}
   Entry(\underline{id}, name, production\_year, type $ \rightarrow $ Type) \\
   Person(\underline{id}, name, task $ \rightarrow $ Task) \\
   Type(\underline{id}, name, description) \\
   Task(\underline{id}, name) \\
   participates(person $ \rightarrow $ Person, task $ \rightarrow $ Task, Name)
   \aufgabe{2}
   \begin{enumerate}
   \item \textbf{Entry} \\
\begin{lstlisting}
CREATE TABLE Entry (
	id INTEGER,
  name VARCHAR(255),
	production_year DATE,
	type INTEGER REFERENCES Type
)
\end{lstlisting}
    \item \textbf{Person} \\
\begin{lstlisting}
CREATE TABLE Person (
	id INTEGER, 
	name VARCHAR(255), 
	task INTEGER REFERENCES Task
) 
    \end{lstlisting}
    \item \textbf{Type} \\
\begin{lstlisting}
CREATE TABLE Type (
	id INTEGER, 
	name VARCHAR(255), 
	description VARCHAR(255)
) 
\end{lstlisting}
    \item \textbf{Task} \\
\begin{lstlisting}
CREATE TABLE Task (
	id INTEGER, 
	name VARCHAR(255)
) 
\end{lstlisting}
    \item \textbf{participates} \\
\begin{lstlisting}
CREATE TABLE participates (
	id INTEGER NOT NULL REFERENCES Person,
	id INTEGER NOT NULL REFERENCES Task,
	name VARCHAR(255) REFERENCES Task
)
\end{lstlisting}	
    \end{enumerate}
\aufgabe{3}

\begin{enumerate}[\textbf{Anfrage} 1:]

\item Finde den \textit{title} und \textit{type} aller \textit{Movies} seit 2004. 
\begin{lstlisting} 
SELECT TITLE_TITLE, TITLE_TYPE
FROM IMDB.MOVIES
WHERE TITLE_YEAR >= '2004';
\end{lstlisting}

\item Finde die Name aller \textit{Actors} (und \textit{Actresses}), die in mindestens einem der Filme aus Anfrage 1 mitgespielt haben. Gib jeden Namen nur einmal aus!
\begin{lstlisting}
SELECT DISTINCT NAME_NAME
FROM IMDB.ACTORS a, IMDB.MOVIES m
WHERE a.TITLE_ID = m.TITLE_ID
AND m.TITLE_YEAR >='2004';
\end{lstlisting}

\item Finde heraus, wie viele \textit{Movies} es seit 2004 gegeben hat (die Anzahl)!
\begin{lstlisting}
SELECT COUNT(*)
FROM IMDB.MOVIES
WHERE TITLE_YEAR >='2004'
AND TITLE_TYPE='film';
\end{lstlisting}

\item Finde die maximale und durchschnittliche Anzahl von \textit{Movies}, in denen \textit{Actors} (und \textit{Actresses}) seit 2004 mitgespielt haben. \textit{Actors} (und \textit{Actresses}), die nur in einem \textit{Movie} mitgespielt haben, sollen nicht mit betrachtet werden!
\begin{lstlisting}
\end{lstlisting}

\item Finde die \textit{names} aller \textit{Actors} (und \textit{Actresses}), die in mindestens 30 \textit{Movies} (nicht \textit{Serien}) seit 2004 mitgespielt haben und die Anzahl der \textit{Movies} in denen sie mitgespielt haben, absteigend sortiert nach der Anzahl der \textit{title} und ihre \textit{names} in alphabetischer Reihenfolge.
\begin{lstlisting}
\end{lstlisting}
\end{enumerate}

\aufgabe{4}
\begin{lstlisting}
INSERT INTO Entry (Name, production_year)
\end{lstlisting}
\end{document}

