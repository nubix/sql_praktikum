\documentclass[11pt,a4paper,DIV=9]{scrartcl}
\usepackage{pgf}
\usepackage{tikz}
\usetikzlibrary{arrows,automata,positioning,shadows}
\usepackage{ngerman}
\usepackage[utf8]{inputenc}
\usepackage{amsmath,amssymb}
\usetikzlibrary{shapes,snakes}
% Schriftart ändern
\renewcommand{\rmdefault}{ppl}
%Möglichkeit zur Änderung von Überschriften
\usepackage{sectsty}
%Überschrift \section uandern
\definecolor{blue}{RGB}{76 , 92, 153}
\allsectionsfont{\color{blue}}
\paragraphfont{\color{blue}}

%Variable Blattnummer
\newcommand{\blatt}[1]{
  \newcommand{\blattnr}{#1}
}
%Aufgabe und Aufgabenteil definieren
\newcounter{temp}
\newcommand{\aufgabe}[1]{
  \setcounter{temp}{\value{subsection}}
  \setcounter{subsection}{#1}
  \addtocounter{subsection}{-1}
  \subsection{Aufgabe}
  \setcounter{subsection}{\value{temp}}
}
\newcommand{\teil}[2][]{
  \subsubsection*{#2) #1}
}
\renewcommand{\author}[1]{\renewcommand{\author}{#1}}
\renewcommand{\title}[1]{\renewcommand{\title}{#1}}
\newcommand{\makehomeworktitle}{
  \begin{minipage}[t]{6.5cm}
    \sf{\author}
  \end{minipage}
  \begin{minipage}[t]{6.5cm}
    \begin{flushright}
      \sf{\title\\\today}
    \end{flushright}
  \end{minipage}
  \\[0.2cm]
  \begin{center}
    \sf{
      \color{blue}{
        \LARGE{Dokumentation \blattnr}
      }
    }
  \end{center}
  \vspace{0.1cm}
}

%%%%%%%%%%%%%%%%%%%%%%%%
%%% Statisch
\author{{[}4131658{]} Jan Germann \\{[}4054962{]} Christian Ratz\\Übungsgruppe 1}
\title{SQL Praktikum}

%%% Auf jedes Hausaufgabenblatt anpassen
\blatt{1}
%%%%%%%%%%%%%%%%%%%%%%%%
\setcounter{section}{\blattnr}
\begin{document}
\makehomeworktitle

\teil{(Einleitung}
Unsere Dokumentation erkl\"art unser Modell zu dem communitybasierten Streamingdienst f\"ur Musik FIRST.FM.
In der Dokumentation gilt folgende Formatierung: \\
\\\textbf{Klasse}:						Name der Klasse [\textbf{fett}] 
\\\texttt{Attribut}: 					Attribut einer Klasse [- davor, normal]
\\\texttt{\textsc{\underline{Prim\"arschl\"ussel}}}:		Attribut ist ein Prim\"arschl\"ussel [* davor]
\\\texttt{\textsc{Fremdschl\"ussel}}: 			Attribut ist ein Fremdschl\"ussel [* davor]
\\\texttt{\textit{Abgeleitetes Attribut}}:		Attribut ist wird abgeleitet [/ davor, \textit{kursiv}]
\\\\Anmerkung: Am Ende der Dokumentation ist das komplette Klassendiagramm nochmal abgebildet. \\
\teil{(Detail}

Die Klasse \textbf{User} hat eine eindeutige \texttt{\textsc{\underline{id}}} sowie einen \texttt{loginname} und eine \texttt{emailadress}. Des Weiteren wird ein Datum gespeichert wann der Benutzer zuletzt online gewesen ist (\texttt{lastseen}) und es wird anderen Nutzern angezeigt ob der Benutzer zurzeit online ist oder nicht (\texttt{online}).
\\\\
Von jedem Nutzer wird ein Verlauf der bereits geh\"orten Songs gespeichert. Dies erfolgt mit Hilfe von \textbf{UserToPlayed}. Dabei wird einmal der Benutzer (\texttt{userid}) und der jeweilige Song (\texttt{songid}) mit einem zugeh\"origen Zeitstempel (\texttt{timestamp}) gespeichert.
\\\\
Jeder Nutzer hat die M\"oglichkeit anderen Benutzern einen Song zu empfehlen. Dies wird durch \textbf{Recommendation} gew\"ahrleistet. Diese baut sich einmal aus dem Empfehlenden (\texttt{recommenderid}), dem Empfehlsempf\"anger (\texttt{recipientid}) und den empfohlenen Song auf (\texttt{songid}). 
\\\\
Man kann au{\ss}erdem neue Songs hochladen (\textbf{Upload}), dabei ist man entweder ein einfacher Benutzer von FIRST.FM, oder der K\"unstler des hochgeladenen Songs (\textbf{ArtistOfSong}) oder Mitglied in einer Band (\textbf{UserToBand}) dessen Song man hochgeladen hat. \\ Jeder K\"unstler in einer Band spielt ein Instrument (\textbf{Instrument}), was auch seine Stimme sein kann. Jedes Instrument hat eine identifizierende Nummer (\texttt{id}) und einen Namen (\texttt{name}).
\\\\
Der hochgeladene Song l\"asst sich von anderen Nutzern ebenfalls bewerten (\textbf{Songrating}), ab einer gewissen Bewertung erh\"alt der Song erh\"alt der Song einen Goldstatus (\underline{4,5 bei \"uber 10000} Bewertungen) oder einen Platinstatus (\underline{4,6 bei \"uber 100000} Bewertungen).
\\\\
Jeder hochgeladene Song (\textbf{Song}) besitzt eine \texttt{\textsc{\underline{id}}}, einen Songtitel (\texttt{title}), ein Genre (\texttt{genreid}) dem es angeh\"ort, ein Datum wann der Song ver\"offentlicht wurde (\texttt{releasedate}), einen Status (\texttt{\textit{status}}) der abgeleitet wird, da er von der Bewertung abh\"angig ist und eine Adresse unter der der Song erreichbar ist (\texttt{url}). Jedes Genre (\textbf{Genre}) besitzt einen Namen (\texttt{name}) sowie einen unterscheidbaren Schl\"ussel (\texttt{\textsc{\underline{genreid}}}). 
\\\\
Des Weiteren besteht die M\"oglichkeit andere Benutzer in seine Freundesliste aufzunehmen (\textbf{Friendslist}), diese lassen sich auch kommentieren (\textbf{CommentToUser}). Dabei wird erstmal der kommentierende Benutzer (\texttt{userid}), der kommentierte Benutzer (\texttt{commenteduserid}), der Inhalt des Kommentars (\texttt{content}) und ebenfalls ein Zeitpunkt des Kommentars (\texttt{timestamp}) gesichert. Genauso lassen sich Bands (\textbf{CommentToBand}) oder auch Songs (\textbf{CommentToSong}) kommentieren. Dabei wird der kommentierte Benutzer durch die jeweilige Band (\texttt{\textsc{bandid}}) oder den jeweiligen Song (\texttt{\textsc{songid}}) getauscht.
\\

\end{document}