\documentclass[11pt,a4paper,DIV=9]{scrartcl}

\usepackage{pgf}
\usepackage{tikz}
\usetikzlibrary{arrows,automata,positioning,shadows}
\usepackage{ngerman}
\usepackage[utf8]{inputenc}
\usepackage{amsmath,amssymb}
\usetikzlibrary{shapes,snakes}
% Schriftart ändern
\renewcommand{\rmdefault}{ppl}
%Möglichkeit zur Änderung von Überschriften
\usepackage{sectsty}
%Überschrift \section uandern
\definecolor{blue}{RGB}{76 , 92, 153}
\allsectionsfont{\color{blue}}
\paragraphfont{\color{blue}}

%Variable Blattnummer
\newcommand{\blatt}[1]{
  \newcommand{\blattnr}{#1}
}
%Aufgabe und Aufgabenteil definieren
\newcounter{temp}
\newcommand{\aufgabe}[1]{
  \setcounter{temp}{\value{subsection}}
  \setcounter{subsection}{#1}
  \addtocounter{subsection}{-1}
  \subsection{Aufgabe}
  \setcounter{subsection}{\value{temp}}
}
\newcommand{\teil}[2][]{
  \subsubsection*{#2) #1}
}
\renewcommand{\author}[1]{\renewcommand{\author}{#1}}
\renewcommand{\title}[1]{\renewcommand{\title}{#1}}
\newcommand{\makehomeworktitle}{
  \begin{minipage}[t]{6.5cm}
    \sf{\author}
  \end{minipage}
  \begin{minipage}[t]{6.5cm}
    \begin{flushright}
      \sf{\title\\\today}
    \end{flushright}
  \end{minipage}
  \\[0.2cm]
  \begin{center}
    \sf{
      \color{blue}{
        \LARGE{Dokumentation \blattnr}
      }
    }
  \end{center}
  \vspace{0.1cm}
}

%%%%%%%%%%%%%%%%%%%%%%%%
%%% Statisch
\author{{[}4131658{]} Jan Germann \\{[}4054962{]} Christian Ratz\\Übungsgruppe 1}
\title{SQL Praktikum}

%%% Auf jedes Hausaufgabenblatt anpassen
\blatt{1}
%%%%%%%%%%%%%%%%%%%%%%%%
\begin{document}
\makehomeworktitle
\tableofcontents
\newpage
\section{Einleitung}
  \subsection{Allgemeines}
    In dieser Dokumentation werden die Besonderheiten des Klassendiagramms, für das communitybasierte Musik-Streamingportal \textsc{first.fm}, erläutert. Bei dem Musik-Streamingportal \textsc{first.fm}, haben Benutzer die Möglichkeit selbstkomponierte Musik zu veröffentlichen, Musik anderer Benutzer oder Bands zu hören sowie selbst einer Band beizutreten.\\
    Am Ende dieser Dokumentation ist, zur besseren Übersicht, das komplette Klassendiagramm noch einmal in seiner Vollständigkeit zu finden.

  \subsection{Formatierung} 
    In der Dokumentation sowie im dazugehörigen Diagramm gilt die folgende Formatierung.

    \begin{description}
      \item [Klassennamen] \hfill \\
        Der Name einer Klasse ist im Regelfall \textbf{fett} formatiert.
      \item [Attribute] \hfill \\
        Attribute einer Klasse sind folgendermaßen gekennzeichnet
        \begin{itemize}
          \item[-] Reguläres Attribut
          \item[*] Attribut ist Teil des Primärschlüssel
          \item[/] Abgeleitetes Attribut
          \item[+] Fremdschlüssel
        \end{itemize}
      \item[Relationentypen] \hfill
      \begin{itemize}
        \item Reguläre Relationen zwischen Klassen sind durch eine einfache schwarze Linie gekennzeichnet.
        \item Sogenannte \textit{XOR}-Relationen besitzen an der ausgehenden Klasse ein Rautesymbol ($\blacklozenge$). Hierbei darf die Klasse, für gespeicherten Entität, nur mit einer einzigen der Klassen gleichzeit in Relation stehen.
      \end{itemize}
    \end{description}

\section{Klassen}
  \subsection{User}
    Eine Entität der Klasse \textbf{User} wird aus Speicherplatzgründen durch einen Surrogatschlüssel (*id) identifiziert. Ein anderer Schlüsselkandidat ist hierbei der -loginname, dieser ist ebenfalls einmalig. Als weitere Besonderheit ist hierbei das Attribut -online zu sehen, welches \texttt{1} ist wenn der Benutzer online ist und \texttt{0} sonst. Der Onlinestatus lässt sich leider nicht ohne weiteres aus dem Attribut -last\_seen ableiten. Ein User ist nach dem Abmelden sofort Offline ist, allerdings auch sobald die Differenz zwischen -last\_seen und der Jetztzeit größer als 15 Minuten ist.
  \subsection{Band}
    Eine \textbf{Band} besteht aus mindestens einem Usern. Dies mag auf anhieb nicht Sinnvoll erscheinen. Allerdings kann ein Benutzer auch Mitglied einer Band sein ohne, dass der Rest der Band bei \textsc{first.fm} einen Account haben muss. 
  \subsection{Bandmember}
    Die Klasse \textbf{Bandmember} soll es den Benutzern einer Band ermöglichen, Mitglieder einer Band hinzuzufügen welche nicht über einen Benutzeraccount bei \textsc{first.fm} verfügen. Um die Zugehörigkeit zu einer bestimmten Band zu definieren verfügen diese Entitäten über den Fremdschlüssel +band\_id. Desweiteren verfügen diese Entitäten über den Fremdschlüssel +instrument\_id, welche das von Ihnen in der Band gespiele Instrument definiert.
  \subsection{Song}
  \subsection{Album}
  \subsection{Label}
  \subsection{Genre}
  \subsection{Comment}
  \subsection{Comment2User}
  \subsection{Comment2Song}
  \subsection{Comment2Band}
  \subsection{Instrument}
  \subsection{Upload}
  \subsection{Recomendation}
  \subsection{Songrating}
  \subsection{Friendslist}
  \subsection{User2Played}
  \subsection{Artist2Song}
  \subsection{User2Band}
  \subsection{Band2Song}
  \subsection{Song2Genre}
  \subsection{Song2Album}
  \subsection{Album2Label}



\section{Klassendiagramm}
\includegraphics[page=1, angle=90,trim=0cm 0cm 1.1cm 1cm, clip=true, scale=1.1]{Diagram1}
\end{document}