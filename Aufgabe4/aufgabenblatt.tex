\documentclass[11pt,a4paper,DIV=9]{scrartcl}
\usepackage{pgf}
\usepackage{tikz}
\usetikzlibrary{arrows,automata,positioning,shadows}
\usepackage{ngerman}
\usepackage[utf8]{inputenc}
\usepackage{amsmath,amssymb}
\usepackage{tikz-er2}
\usepackage{enumerate}
\usepackage{listings}
\usetikzlibrary{shapes,snakes}
% Schriftart ändern
\renewcommand{\rmdefault}{ppl}
%Möglichkeit zur Änderung von Überschriften
\usepackage{sectsty}
%Überschrift \section uandern
\definecolor{blue}{RGB}{76 , 92, 153}
\allsectionsfont{\color{blue}}
\paragraphfont{\color{blue}}

%Variable Blattnummer
\newcommand{\blatt}[1]{
  \newcommand{\blattnr}{#1}
}
%Aufgabe und Aufgabenteil definieren
\newcounter{temp}
\newcommand{\aufgabe}[1]{
  \setcounter{temp}{\value{subsection}}
  \setcounter{subsection}{#1}
  \addtocounter{subsection}{-1}
  \subsection{Aufgabe}
  \setcounter{subsection}{\value{temp}}
}
\newcommand{\teil}[2][]{
  \subsubsection*{#2) #1}
}
\renewcommand{\author}[1]{\renewcommand{\author}{#1}}
\renewcommand{\title}[1]{\renewcommand{\title}{#1}}
\newcommand{\makehomeworktitle}{
  \begin{minipage}[t]{6.5cm}
    \sf{\author}
  \end{minipage}
  \begin{minipage}[t]{6.5cm}
    \begin{flushright}
      \sf{\title\\\today}
    \end{flushright}
  \end{minipage}
  \\[0.2cm]
  \begin{center}
    \sf{
      \color{blue}{
        \LARGE{Three degrees of Heath Ledger \blattnr}
      }
    }
  \end{center}
  \vspace{0.1cm}
}

%%%%%%%%%%%%%%%%%%%%%%%%
%%% Statisch
\author{{[}4131658{]} Jan Germann \\{[}4054962{]} Christian Ratz\\Übungsgruppe 1}
\title{SQL Praktikum}

%%% Auf jedes Hausaufgabenblatt anpassen
\blatt{4}
%%%%%%%%%%%%%%%%%%%%%%%%
\setcounter{section}{\blattnr}

\definecolor{dkgreen}{rgb}{0,0.6,0}
\definecolor{gray}{rgb}{0.5,0.5,0.5}
\definecolor{mauve}{rgb}{0.58,0,0.82}

\lstset{ 
  basicstyle=\footnotesize\ttfamily,
  language=sql,                % the language of the code
  numbers=left,
  numberstyle=\tiny\color{gray},
  keywordstyle=\color{blue},          % keyword style
  commentstyle=\color{dkgreen},       % comment style
  stringstyle=\color{mauve}}

\begin{document}
\makehomeworktitle
 \aufgabe{1}
 \begin{enumerate}[a.]
 \item\
 SELECT-Anfrage auf den \textit{database catalog} um alle IMDB-Schema Tabellen auszugeben: 
 \begin{lstlisting}
 SELECT TABNAME as Tabellen
 FROM SYSCAT.TABLES 
 WHERE TABSCHEMA = 'IMDB'  
 \end{lstlisting}
 
 \item\ 
 SELECT-Anfrage auf den \textit{database catalog} um die Struktur der IMDB.MOVIES-Tabelle auszugeben:
 \begin{lstlisting}
 SELECT COLNAME, TYPENAME, LENGTH 
 FROM SYSCAT.COLUMNS 
 WHERE TABSCHEMA = 'IMDB' 
 AND TABNAME = 'MOVIES' ORDER BY COLNO
 \end{lstlisting} 
 
 \item\
 \textbf{IMDB.MOVIES}-Struktur manuell als \textbf{tv\_movies\_tmp} kopiert: 
 \begin{lstlisting}
CREATE TABLE tv_movies_tmp (
TITLE_ID varchar(400),
TITLE_TYPE varchar(100),
TITLE_TITLE varchar(400),
TITLE_YEAR integer,
TITLE_NUMERAL integer,
TITLE_SUSPENDED decimal,
TITLE_ATTRIBUTES varchar(400),
EPISODE_SERIES varchar(400),
EPISODE_TITLE varchar(400),
EPISODE_TITLE_NUMERAL integer,
EPISODE_SEASON integer,
EPISODE_EPISODE integer,
EPISODE_DATE date,
YEAR_RANGE varchar(100));
 \end{lstlisting}
 \end{enumerate}
 \aufgabe{2}
 Struktur einer Tabelle mit 
 \begin{lstlisting} 
 CREATE TABLE LIKE 
 \end{lstlisting}
 kopieren: (in diesem Fall die Struktur von \textbf{IMDB.ACTORS})
 \begin{lstlisting}
 CREATE TABLE tv_actors_tmp LIKE IMDB.ACTORS
 \end{lstlisting}
 \aufgabe{3}
 \begin{enumerate}[a.]
 \item
 Tabellen die \"uberhaupt keine NULL-Werte enthalten: \\\\
 \textbf{IMDB.MOVIES}: \\
 TITLE\_ID \\
 TITLE\_TYPE \\
 TITLE\_SUSPENDED \\\\
 \textbf{IMDB.ACTORS}: \\
 NAME\_NAME \\
 TITLE\_ID \\
 NAME\_ID \\
 \item
 NOT NULL CONSTRAINT zu \textbf{tv\_movies\_tmp} und zu \textbf{tv\_actors\_tmp} hinzuf\"ugen: 
 \textbf{tv\_movies\_tmp}
 \begin{lstlisting}
 ALTER TABLE tv_movies_tmp ALTER COLUMN title_id SET NOT NULL
 ALTER TABLE tv_movies_tmp ALTER COLUMN title_type SET NOT NULL
 ALTER TABLE tv_movies_tmp ALTER COLUMN title_suspended SET NOT NULL
 \end{lstlisting}
 \textbf{tv\_actors\_tmp}
 \begin{lstlisting}
 ALTER TABLE tv_actors_tmp ALTER COLUMN name_id SET NOT NULL
 ALTER TABLE tv_actors_tmp ALTER COLUMN name_name SET NOT NULL
 ALTER TABLE tv_actors_tmp ALTER COLUMN title_id SET NOT NULL
 \end{lstlisting} 
 \end{enumerate}
 \aufgabe{4}
 Struktur von \textbf{tv\_movies\_tmp} und \textbf{tv\_actors\_tmp} kopieren und zwei neue Tabellen \textbf{tv\_movies} und \textbf{tv\_actors} mit derselben Struktur erstellen: 
 \begin{lstlisting}
 CREATE TABLE tv_movies LIKE tv_movies_tmp
 CREATE TABLE tv_actors LIKE tv_actors_tmp
 \end{lstlisting}
 \aufgabe{5}
 Passende PRIMARY KEYS und FOREIGN KEYS einf\"ugen:
 \begin{lstlisting}
 ALTER TABLE TV_MOVIES ADD PRIMARY KEY (TITLE_ID, TITLE_TYPE);
 ALTER TABLE TV_ACTORS ADD PRIMARY KEY (NAME_ID, NAME_NAME); 
 ALTER TABLE TV_ACTORS ADD UNIQUE (TITLE_ID);
 ALTER TABLE TV_MOVIES ADD UNIQUE (TITLE_ID);
 ALTER TABLE TV_ACTORS ADD FOREIGN KEY (TITLE_ID) REFERENCES TV_MOVIES (TITLE_ID);
 ALTER TABLE TV_MOVIES ADD FOREIGN KEY (TITLE_ID) REFERENCES TV_ACTORS (TITLE_ID);
 \end{lstlisting}
 \aufgabe{6}
 \textbf{tv\_movies} und \textbf{tv\_actors} hit entsprechenden Daten f\"ullen:
 Nur \textit{TV movies} aus dem Jahr \textit{2008} und alle ihre \textit{actors} und \textit{actresses} eintragen:
 \begin{lstlisting}
INSERT INTO DBLAB01.TV_MOVIES (TITLE_ID, TITLE_TYPE, TITLE_TITLE, TITLE_YEAR, 
TITLE_NUMERAL, TITLE_NUMERAL, TITLE_SUSPENDED, TITLE_ATTRIBUTES, 
EPISODE_SERIES, EPISODE_TITLE, EPISODE_TITLE_NUMERAL, EPISODE_SEASON, 
EPISODE_EPISODE, EPISODE_DATE, YEAR_RANGE)
 (SELECT DISTINCT TITLE_ID, TITLE_TYPE, TITLE_TITLE, TITLE_YEAR, 
 TITLE_NUMERAL, TITLE_NUMERAL, TITLE_SUSPENDED, TITLE_ATTRIBUTES, 
 EPISODE_SERIES, EPISODE_TITLE, EPISODE_TITLE_NUMERAL, EPISODE_SEASON, 
 EPISODE_EPISODE, EPISODE_DATE, YEAR_RANGE
 FROM IMDB.MOVIES AS m, IMDB.ACTORS AS a
 WHERE m.TITLE_ID = a.TITLE_ID
 AND TITLE_TYPE = 'TV movie'
 AND TITLE_YEAR = '2008');
 
INSERT INTO DBLAB01.TV_MOVIES (TITLE_ID, TITLE_TYPE, TITLE_TITLE, TITLE_YEAR, 
TITLE_NUMERAL, TITLE_NUMERAL, TITLE_SUSPENDED, TITLE_ATTRIBUTES, 
EPISODE_SERIES, EPISODE_TITLE, EPISODE_TITLE_NUMERAL, EPISODE_SEASON, 
EPISODE_EPISODE, EPISODE_DATE, YEAR_RANGE)
 (SELECT DISTINCT TITLE_ID, TITLE_TYPE, TITLE_TITLE, TITLE_YEAR, 
 TITLE_NUMERAL, TITLE_NUMERAL, TITLE_SUSPENDED, TITLE_ATTRIBUTES, 
 EPISODE_SERIES, EPISODE_TITLE, EPISODE_TITLE_NUMERAL, EPISODE_SEASON, 
 EPISODE_EPISODE, EPISODE_DATE, YEAR_RANGE
 FROM IMDB.MOVIES AS m, IMDB.ACTRESSES AS a
 WHERE m.TITLE_ID = a.TITLE_ID
 AND TITLE_TYPE = 'TV movie'
 AND TITLE_YEAR = '2008');
 
 INSERT INTO DBLAB01.TV_ACTORS (NAME_ID, NAME_NAME, NAME_NUMERAL, TITLE_ID, 
 CHARACTERS, CREDIT_ORDER_NUMBER, CREDIT_ATTRIBUTES)
 (SELECT DISTINCT NAME_ID, NAME_NAME, NAME_NUMERAL, TITLE_ID, 
 CHARACTERS, CREDIT_ORDER_NUMBER, CREDIT_ATTRIBUTES
 FROM IMDB.ACTORS AS a, IMDB.MOVIES AS m
 WHERE a.TITLE_ID = m.TITLE_ID
 AND TITLE_TYPE = 'TV movie'
 AND TITLE_YEAR = '2008');
 
  INSERT INTO DBLAB01.TV_ACTORS (NAME_ID, NAME_NAME, NAME_NUMERAL, TITLE_ID, 
  CHARACTERS, CREDIT_ORDER_NUMBER, CREDIT_ATTRIBUTES)
 (SELECT DISTINCT NAME_ID, NAME_NAME, NAME_NUMERAL, TITLE_ID, 
 CHARACTERS, CREDIT_ORDER_NUMBER, CREDIT_ATTRIBUTES
 FROM IMDB.ACTRESSES AS a, IMDB.MOVIES AS m
 WHERE a.TITLE_ID = m.TITLE_ID
 AND TITLE_TYPE = 'TV movie'
 AND TITLE_YEAR = '2008');
 
 \end{lstlisting}
 \aufgabe{7}
 \textit{materialized view} namens \textit{actor\_co\_occurrences} erstellen, die \textit{tv\_movies} und  \textit{tv\_actors} basiert. Folgende Signatur: 
 \begin{lstlisting}
 actor_co_occurrences(actor1, actor2, movie)
 \end{lstlisting}
 \aufgabe{8}
 F\"ur \textit{actor1}, \textit{actor2} und \textit{movie} aus der \textit{actor\_co\_occurrences} einen eigenen Index anlegen.
 \aufgabe{9}
 Alle Schauspieler, die eine \textit{Heath Ledger Zahl} von maximal 3 haben. Gib ihre Namen und ihre \textit{Heath Ledger Zahl} aus. Das benutzte SELECT Statement soll \textbf{nicht} rekursiv sein!
 \aufgabe{10}
 SELECT-Anfrage aus Aufgabe 9 (rekursive Version)
\end{document}

